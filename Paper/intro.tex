\documentclass[a4paper,12pt,abstracton,titlepage]{scrartcl}

%\usepackage[T1]{fontenc}
%\usepackage{lmodern}
%\usepackage[utf8]{inputenc}
%\usepackage[english]{babel}


\usepackage{graphicx}
\usepackage{dcolumn}
\usepackage{bm}
\usepackage[utf8]{inputenc}
\usepackage{amsmath}
\usepackage[english]{babel}
\usepackage{graphicx}
\usepackage[colorinlistoftodos]{todonotes}
\usepackage{tikz-3dplot}
\usepackage{adjustbox}
\usepackage{amsfonts}
\usepackage{pgfplots}
\usepackage{url}
\usepackage{algpseudocode}
\usepackage{algorithm}
\usepackage{siunitx}
\usepackage{subfig}

\pagenumbering{gobble}

\titlehead{Värmdö Gymnasium}
\title{On the properties of projectile motion and quadratic air resistance}

\author{Simon Halvdansson}

\publishers{
	\normalfont\normalsize
	\parbox{0.8\linewidth}{
		The equations of motion for normal projectile motion are derived. Some properties of such motion are defined. Numerical approximations and analytical solutions to the effects of one dimensional quadratic air resistance are made and compared to experimental results with small error. Analytical approximations to two dimensional quadratic air resistance are made using a model which separates low, high and split angle trajectories. The equations of motion for these are derived, plotted and compared to numerical solutions to the original differential equation.
	}
}

\begin{document}
	
	\maketitle
	\clearpage
	
	\contentsline {section}{\numberline {1}Introduction}{1}{}
	\contentsline {section}{\numberline {2}Trajectory}{1}{}
	\contentsline {subsection}{\numberline {2.1}Acceleration}{2}{}
	\contentsline {subsection}{\numberline {2.2}Velocity}{2}{}
	\contentsline {subsection}{\numberline {2.3}Position}{2}{}
	\contentsline {subsection}{\numberline {2.4}Parabola}{2}{}
	\contentsline {section}{\numberline {3}Properties of a projectile parabola}{2}{}
	\contentsline {subsection}{\numberline {3.1}Time to highest point}{2}{}
	\contentsline {subsection}{\numberline {3.2}Highest point}{3}{}
	\contentsline {subsection}{\numberline {3.3}Range of projectile}{3}{}
	\contentsline {subsection}{\numberline {3.4}Optimal angle for maximum range}{3}{}
	\contentsline {subsection}{\numberline {3.5}Velocity at a given time}{3}{}
	\contentsline {subsection}{\numberline {3.6}Angle of impact}{3}{}
	\contentsline {section}{\numberline {4}Choosing initial values to match result values}{3}{}
	\contentsline {subsection}{\numberline {4.1}Delimitations}{3}{}
	\contentsline {subsection}{\numberline {4.2}Specific projectile range}{4}{}
	\contentsline {subsection}{\numberline {4.3}Hitting a specific coordinate}{4}{}
	\contentsline {subsubsection}{\numberline {4.3.1}Predetermined speed}{4}{}
	\contentsline {subsubsection}{\numberline {4.3.2}Predetermined angle}{4}{}
	\contentsline {section}{\numberline {5}Longest parabola starting at height}{5}{}
	\contentsline {subsection}{\numberline {5.1}Expression for range}{5}{}
	\contentsline {subsection}{\numberline {5.2}Algorithm}{5}{}
	\contentsline {section}{\numberline {6}Computational approach to air resistance}{6}{}
	\contentsline {subsection}{\numberline {6.1}Change at every time interval}{6}{}
	\contentsline {subsection}{\numberline {6.2}Algorithm}{7}{}
	\contentsline {section}{\numberline {7}Analytical solution to one-dimensional air resistance}{7}{}
	\contentsline {subsection}{\numberline {7.1}Derivation}{7}{}
	\contentsline {subsubsection}{\numberline {7.1.1}Position as a function of time}{7}{}
	\contentsline {subsubsection}{\numberline {7.1.2}Determining the duration of the fall}{8}{}
	\contentsline {subsubsection}{\numberline {7.1.3}Terminal velocity}{8}{}
	\contentsline {subsection}{\numberline {7.2}Plots}{8}{}
	\contentsline {subsubsection}{\numberline {7.2.1}Examples}{8}{}
	\contentsline {subsubsection}{\numberline {7.2.2}Analysis}{9}{}
	\contentsline {section}{\numberline {8}Analytical approximations to two-dimensional air resistance}{9}{}
	\contentsline {subsection}{\numberline {8.1}Introduction}{9}{}
	\contentsline {subsection}{\numberline {8.2}Low Angle Trajectory approximation}{10}{}
	\contentsline {subsubsection}{\numberline {8.2.1}Differential equations of motion}{10}{}
	\contentsline {subsubsection}{\numberline {8.2.2}Velocity}{10}{}
	\contentsline {subsubsection}{\numberline {8.2.3}Position}{10}{}
	\contentsline {subsubsection}{\numberline {8.2.4}Equations of motion}{11}{}
	\contentsline {subsection}{\numberline {8.3}High Angle Trajectory approximation}{11}{}
	\contentsline {subsubsection}{\numberline {8.3.1}Differential equations of motion}{11}{}
	\contentsline {subsubsection}{\numberline {8.3.2}Ascending velocity}{11}{}
	\contentsline {subsubsection}{\numberline {8.3.3}Ascending position}{11}{}
	\contentsline {subsubsection}{\numberline {8.3.4}Descending velocity}{12}{}
	\contentsline {subsubsection}{\numberline {8.3.5}Descending position}{12}{}
	\contentsline {subsubsection}{\numberline {8.3.6}Equations of motion}{13}{}
	\contentsline {subsection}{\numberline {8.4}Split Angle Trajectory approximation}{13}{}
	\contentsline {subsubsection}{\numberline {8.4.1}Differential equations of motion}{13}{}
	\contentsline {subsubsection}{\numberline {8.4.2}Equations of motion}{13}{}
	\contentsline {subsection}{\numberline {8.5}Plots}{13}{}
	\contentsline {subsubsection}{\numberline {8.5.1}Low Angle Trajectory}{13}{}
	\contentsline {subsubsection}{\numberline {8.5.2}High Angle Trajectory}{14}{}
	\contentsline {subsubsection}{\numberline {8.5.3}Split Angle Trajectory}{14}{}
	\contentsline {section}{\numberline {9}Experiment}{14}{}
	\contentsline {subsection}{\numberline {9.1}Setup}{14}{}
	\contentsline {subsection}{\numberline {9.2}Results}{15}{}
	\contentsline {subsection}{\numberline {9.3}Analysis}{15}{}
	\contentsline {section}{\numberline {10}Conculsions}{15}{}
	\contentsline {section}{\numberline {}Further reading}{15}{}
	
\end{document}